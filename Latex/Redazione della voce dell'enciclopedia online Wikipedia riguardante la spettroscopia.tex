\documentclass[12pt,a4paper]{report}
\usepackage[italian]{babel}
\usepackage{newlfont}
\usepackage{ragged2e}
\usepackage{csquotes}
\usepackage{hyperref, cleveref}
\usepackage{graphicx, caption, subcaption}

%Bibliografia
\usepackage[style=numeric,sorting=none]{biblatex}
\addbibresource{bibliografia.bib}

\textwidth=450pt\oddsidemargin=0pt

\begin{document}
\begin{titlepage}

\begin{center}
{{\Large{\textsc{Alma Mater Studiorum $\cdot$ Universit\`a di Bologna}}}} 
\rule[0.1cm]{15.8cm}{0.1mm}
\rule[0.5cm]{15.8cm}{0.6mm}
\\\vspace{3mm}

{\small{\bf Scuola di Scienze \\ 
Dipartimento di Fisica e Astronomia\\
Corso di Laurea in Fisica}}

\end{center}

\vspace{23mm}

\begin{center}\
{\LARGE{\bf Redazione della voce dell'enciclopedia online Wikipedia riguardante la Spettroscopia}}\\
\end{center}

\vspace{50mm} \par \noindent

\begin{minipage}[t]{0.47\textwidth}\raggedright 
{\large{\bf Relatore: \vspace{2mm}\\
Prof. Enrico Gianfranco Campari
}}
\end{minipage}
%
\hfill
%
\begin{minipage}[t]{0.47\textwidth}\raggedleft
{\large{\bf Presentata da:
\vspace{2mm}\\
Riccardo Peltretti}}
\end{minipage}

\vspace{40mm}

\begin{center}
Anno Accademico 2023/2024
\end{center}

\end{titlepage}

\begin{abstract}
    La presente tesi si propone di migliorare la voce italiana di Wikipedia sulla spettroscopia, un importante strumento nell'ambito della fisica, chimica e astronomia. Partendo dall'analisi delle attuali voci esistenti su Wikipedia in varie lingue, si evidenziano le lacune presenti nella pagina italiana e si propongono integrazioni basate su contenuti scientifici affidabili e aggiornati.
\end{abstract}
\newpage

\vspace*{\fill}
\begin{flushright}
    Hypoteses non fingo
\end{flushright}
\vspace*{\fill}
\newpage

\tableofcontents
\newpage

\chapter*{Introduzione}

Alan Irwin, nel 1995, propose nel suo libro "Citizen Science: A Study of People, Expertise and Sustainable Development" \cite{irwin1995citizen} una visione innovativa del rapporto tra scienza, cittadini e sviluppo sostenibile. Irwin sostiene che lo sviluppo sostenibile non può essere raggiunto senza coinvolgere attivamente i cittadini nel processo scientifico, riconoscendo e valorizzando la loro conoscenza contestuale. 

Vent'anni dopo, Rick Bonney e colleghi \cite{bonney2015} hanno esaminato l'efficacia pratica di questi progetti nel migliorare la comprensione scientifica e nel coinvolgere il pubblico. Bonney et al. identificano quattro categorie principali di progetti di citizen science: raccolta dati, elaborazione dati, progetti basati su curriculum e scienza comunitaria.

I risultati mostrano che i progetti di raccolta dati sono particolarmente efficaci nel produrre risultati scientifici significativi e nel migliorare le conoscenze specifiche dei partecipanti, sebbene questi ultimi siano generalmente soggetti già interessati alla scienza, spesso provenienti da contesti socio-economici privilegiati, deludendo le speranze di democratizzazione di Irwin.

Un esempio recente di risultato significativo prodotto dalla citizen science viene dalla documentazione degli eventi luminosi transitori (TLE). Nel 2019 Hank Schyma e Paul Smith, due appassionati di fotografia di TLE, si sono organizzati per documentare i fenomeni temporaleschi che causano red sprite, anche noti come fulmini dell'alta atmosfera, al fine di migliorare la loro comprensione del fenomeno per avere più successo nel fotografarli.

Nel corso di questa operazione, hanno ripreso un TLE ancora non documentato, una debole luminescenza verde attorno alle sommità degli sprite più energetici, a cui hanno iniziato a riferisi come GHOST (GreenisH emissions from excited Oxigen in Sprite Tops), ipotzzando che il colore verde fosse dovuto all'eccitazione dell'ossigeno nella mesosfera\cite{tle_documentario}. 

Questa scoperta ha suscitato l'interesse della comunità scientifica, portando ad una campagna di indagine spettroscopica del fenomeno. Un articolo recente pubblicato su Nature mostrato che la persistenza del colore verde è soprattutto dovuta all'eccitazione micropolveri metalliche derivanti dall'ablazione dei meteoriti, composte principalmente da ferro\cite{passas-varo2023spectroscopy}.

Ricercando questo fenomeno, mi sono reso conto dell'inadeguatezza della pagina italiana di Wikipedia sulla spettroscopia. Wikipedia, con la sua natura collaborativa e accessibile, rappresenta una risorsa fondamentale per la divulgazione scientifica e per supportare iniziative di citizen science. La pagina italiana attualmente non copre adeguatamente l'evoluzione storica della spettroscopia e non fornisce una spiegazione della fisica dei fenomeni rilevanti. 

Data l'importanza di Wikipedia come strumento educativo, ho pensato di contribuire proponendo delle integrazioni con l'obbiettivo di rendere la pagina più completa e informativa, migliorando la qualità delle informazioni disponibili, e cercando di stimolare l'interesse e la comprensione della scienza tra un pubblico più ampio.

\chapter{Wikipedia: stora e impatto sulla divulgazione}

\section{Introduzione a Wikipedia}

Wikipedia è un'enciclopedia online gratuita, collaborativa e multilingue, fondata il 15 gennaio 2001 da Jimmy Wales e Larry Sanger \cite{lih2009wikipedia}. Il nome è una combinazione del termine "wiki", che indica un tipo di sito web che permette la creazione e modifica collaborativa dei contenuti, e "encyclopedia", dal greco "paideia" che significa educazione o cultura generale. Il termine "wiki", a sua volta, deriva dalla parola hawaiana "wikiwiki", che significa "veloce" o "rapido". Questa scelta rifletteva l'obiettivo di creare un'enciclopedia che potesse essere rapidamente modificata e aggiornata da chiunque avesse accesso a Internet \cite{cunningham2001wiki}. 

Il primo software wiki, chiamato WikiWikiWeb, fu sviluppato da Ward Cunningham nel 1995 per facilitare la condivisione e l'organizzazione collaborativa delle informazioni \cite{cunningham2001wiki}, Wikipedia sfrutta questo modello per consentire agli utenti di creare e modificare le voci tramite un'interfaccia semplice e intuitiva. La piattaforma è accessibile a chiunque abbia una connessione Internet e, sebbene non sia necessario registrarsi per consultare le voci, creare un account offre ulteriori vantaggi e funzionalità, come gli strumenti di comunicazione tra utenti (talkpage ed email) ed un'area privata di sandbox per fare le prove.

Le voci di Wikipedia sono organizzate in modo ipertestuale, con numerosi link interni che connettono gli argomenti correlati. Questa struttura facilita la navigazione e permette ai lettori di esplorare un argomento in profondità, senza intaccare la fruibilità delle singole voci. Ogni pagina contiene una sezione bibliografica che elenca le fonti utilizzate, garantendo la verificabilità delle informazioni \cite{denning2005wikipedia}.

Wikipedia è mantenuta da una vasta comunità di volontari, noti come Wikipediani, che contribuiscono alla creazione, revisione e manutenzione delle voci. Ci sono diverse categorie di utenti con ruoli specifici, come gli amministratori, che hanno accesso a strumenti avanzati per la gestione del sito, e i revisori, che controllano la qualità dei contenuti. Le modifiche alle voci sono monitorate tramite una cronologia delle revisioni, che registra ogni cambiamento e permette di ripristinare versioni precedenti in caso di errori o vandalismi. Inoltre, le pagine di discussione permettono agli utenti di confrontarsi su come migliorare le voci \cite{reagle2010good}.

\subsection{Principi e obiettivi}

Wikipedia opera secondo una serie di principi fondamentali che ne guidano il funzionamento e ne assicurano l'integrità e la qualità delle informazioni. Questi principi includono la neutralità, la verificabilità e il contenuto libero.

Il principio di neutralità (NPOV - Neutral Point Of View) richiede che tutte le voci siano scritte in modo imparziale, presentando i fatti senza pregiudizi e riflettendo equamente i diversi punti di vista su un argomento \cite{reagle2010good}. La verificabilità è un altro principio cruciale: ogni informazione deve essere supportata da fonti affidabili, come libri accademici, articoli scientifici e pubblicazioni giornalistiche, che possano essere controllate sia dai revisori che dai lettori \cite{denning2005wikipedia}.

Tutti i testi e i media pubblicati su Wikipedia sono disponibili sotto licenze libere, come la Creative Commons Attribution-ShareAlike License (CC BY-SA). Le licenze libere permettono a chiunque di utilizzare, modificare e distribuire i contenuti di Wikipedia, purché venga data attribuzione agli autori originali e le opere derivate siano distribuite con la stessa licenza \cite{jemielniak2014wikipedia}.

Gli utenti di Wikipedia devono seguire una serie di linee guida editoriali che aiutano a mantenere la coerenza e la qualità delle voci, includendo la citazione delle fonti, l'evitare conflitti di interesse e il rispetto delle politiche di copyright. Le voci devono essere scritte in modo chiaro e comprensibile, evitando l'uso di linguaggio tecnico non necessario \cite{reagle2010good}.

\subsection{Struttura e funzionamento generale}

Wikipedia adotta un modello di gestione decentralizzato, in cui la comunità degli utenti svolge un ruolo cruciale nel mantenimento e nella qualità delle voci. A supporto di questa filosofia Gli amministratori, eletti dalla comunità, hanno accesso a strumenti avanzati per la gestione del sito e per la risoluzione delle dispute. Essi possono bloccare utenti problematici, proteggere pagine sensibili e intervenire nei conflitti editoriali \cite{denning2005wikipedia}.

Gli utenti possono contribuire anonimamente o creare un account per ottenere ulteriori funzionalità, come lo storico delle proprie modifiche e la possibilità discutere con altri utenti. Ogni pagina di Wikipedia dispone di una cronologia delle versioni, che registra tutte le modifiche apportate alla pagina, consentendo il ripristino di versioni precedenti in caso di vandalismo o errori \cite{wikiwikiweb}. Inoltre, ad ogni voce dell'enciclopedia è associata una pagina di discussione in cui gli utenti si possono confrontare per decidere le modifiche da apportare, promuovendo un ambiente di collaborazione e miglioramento collettivo \cite{jemielniak2014wikipedia}.

Inoltre, Wikipedia utilizza una serie di bot, ovvero programmi automatizzati, per eseguire compiti ripetitivi come il controllo dei link interrotti, la rimozione di spam e l'aggiornamento delle informazioni. Questi bot contribuiscono a mantenere l'enciclopedia in ordine e a garantire che le informazioni siano sempre aggiornate \cite{history_of_wikis}.

Per supportare queste funzionalità Wikipedia utilizza il software MediaWiki, sviluppato appositamente per Wikipedia e altre iniziative del movimento Wikimedia \cite{denning2005wikipedia}. MediaWiki è un potente software open source, che beneficia di una vasta e attiva comunità di sviluppatori che contribuiscono continuamente al suo miglioramento e aggiornamento. Oltre a Wikipedia, MediaWiki è utilizzato da molte altre organizzazioni e aziende per la gestione collaborativa dei contenuti. Alcune delle aziende e istituzioni che utilizzano MediaWiki includono NASA, che lo impiega per la gestione delle conoscenze interne, Novartis per la documentazione tecnica e la collaborazione scientifica,

\subsection{I cinque pilastri di Wikipedia}

I cinque pilastri di Wikipedia costituiscono i principi fondamentali che guidano il funzionamento e la gestione dell'enciclopedia; sono stati concepiti per mantenere l'integrità, la qualità e la neutralità dei contenuti, garantendo che Wikipedia rimanga una risorsa affidabile e accessibile a tutti.

\paragraph*{1. Wikipedia è un'enciclopedia}
Il primo pilastro stabilisce che Wikipedia è, prima di tutto, un'enciclopedia. Ciò significa che il suo scopo principale è raccogliere e presentare informazioni in modo accurato e verificabile. Wikipedia non è una raccolta indiscriminata di informazioni, un luogo per la promozione personale, una piattaforma di propaganda, o un forum di discussione. Le voci devono essere basate su fonti attendibili e riflettere la conoscenza consolidata, evitando ricerche originali \cite{reagle2010good}.

\paragraph*{2. Wikipedia ha un punto di vista neutrale}
Il secondo pilastro impone che le voci di Wikipedia siano scritte da un punto di vista neutrale (NPOV - Neutral Point Of View). Questo implica che le informazioni devono essere presentate in modo imparziale, rappresentando tutti i punti di vista rilevanti in maniera equa e proporzionata. Gli editori devono evitare di inserire opinioni personali e devono basare le loro affermazioni su fonti verificabili. La neutralità è essenziale per mantenere la credibilità di Wikipedia come fonte di informazione \cite{denning2005wikipedia}.

\paragraph*{3. Wikipedia è libera}
Il terzo pilastro stabilisce che tutto il contenuto di Wikipedia è libero e gratuito. Le voci sono pubblicate sotto licenze libere, come la Creative Commons Attribution-ShareAlike License (CC BY-SA), che permette a chiunque di usare, modificare e distribuire i contenuti, a condizione che venga data attribuzione agli autori originali e che le opere derivate siano rilasciate sotto la stessa licenza \cite{jemielniak2014wikipedia}. Questo principio promuove la diffusione della conoscenza e la collaborazione aperta.

\paragraph*{4. Wikipedia ha un codice di condotta}
Il quarto pilastro riguarda il comportamento degli utenti. Wikipedia incoraggia un ambiente di rispetto e collaborazione. Gli editori devono agire con cortesia e civiltà, evitando attacchi personali e rispettando le opinioni degli altri. Le discussioni sulle modifiche delle voci devono essere condotte in modo costruttivo e rispettoso, facilitando la risoluzione delle dispute e la collaborazione efficace \cite{history_of_wikis}.

\paragraph*{5. Wikipedia non ha regole fisse}
Il quinto pilastro stabilisce che Wikipedia non ha regole fisse oltre i cinque pilastri stessi. Le linee guida e le politiche possono evolvere con il tempo per adattarsi alle nuove esigenze della comunità e del progetto. La flessibilità è fondamentale per permettere a Wikipedia di crescere e migliorare continuamente. Gli utenti sono incoraggiati a essere audaci nelle modifiche, ma sempre con la consapevolezza di rispettare i principi fondamentali \cite{reagle2010good}.

\section{Storia di Wikipedia}

\subsection{Origini}

Wikipedia è stata fondata nel 2001 da Jimmy Wales e Larry Sanger. L'idea di creare un'enciclopedia online libera e collaborativa nacque dal progetto Nupedia, un'enciclopedia online scritta e revisionata da esperti di settore. Nupedia, fondata da Jimmy Wales, era un progetto ambizioso ma soffriva di lentezza nel processo di revisione, che limitava la velocità di ampliamento del contenuto \cite{lih2009wikipedia}.

Per superare questi limiti, Larry Sanger, che lavorava come redattore capo di Nupedia, propose di utilizzare il modello wiki per accelerare il processo di creazione e revisione dei contenuti dell'enciclopedia. Permettendo agli utenti di collaborare direttamente nella stesura delle voci, con la possibilità di correggerle e migliorarne continuamente i contenuti, wikipedia sarebbe stata nelle condizioni di sostenere una crescita organica rapidissima\cite{reagle2010good}.

Inizialmente inquadrata come un esperimento a supporto di Nupedia, il 15 gennaio 2001, Wikipedia fu lanciata ufficialmente come un progetto a se stante e grazie alla sua natura aperta e collaborativa, Wikipedia crebbe rapidamente, attirando contributi da parte di utenti di tutto il mondo. Entro il primo anno, Wikipedia aveva già raccolto migliaia di voci in diverse lingue \cite{jemielniak2014wikipedia}.

La filosofia alla base di Wikipedia era quella di creare una risorsa di conoscenza libera e accessibile a tutti, basata sui principi di neutralità, verificabilità e contenuto libero. Questo approccio democratizzava l'accesso alla conoscenza e permetteva la creazione di contenuti in modo collaborativo, sfruttando l'intelligenza collettiva della comunità globale \cite{denning2005wikipedia}. La nascita di Wikipedia rappresentò un momento cruciale nella storia della diffusione della conoscenza, divenendo rapidamente uno dei principali punti di riferimento per l'informazione online.

\subsection{Evoluzione e crescita nel tempo}

Dalla sua fondazione nel 2001, Wikipedia ha sostenuto una crescita inarrestabile, figura \cref{fig:wikiPagesYear}, diventando una delle principali fonti di informazione su Internet. Nei primi anni, il progetto attirò rapidamente l'attenzione di utenti di tutto il mondo, incrementando il numero di voci in maniera esponenziale \cite{lih2009wikipedia}.

Nel 2003, fu creata la Wikimedia Foundation, un'organizzazione senza scopo di lucro, per supportare Wikipedia e i progetti correlati. La fondazione si occupa della raccolta di fondi, del supporto tecnico e della promozione della missione di Wikipedia di fornire contenuti liberi e accessibili a tutti \cite{reagle2010good}. La creazione della Wikimedia Foundation ha permesso a Wikipedia di ottenere maggiore stabilità e risorse per continuare a crescere.

Nel corso degli anni, Wikipedia ha introdotto numerose funzionalità e miglioramenti tecnici. Ad esempio, nel 2004 fu implementato il sistema di categorie, che permette di organizzare le voci in gruppi tematici, facilitando la navigazione e la ricerca delle informazioni \cite{jemielniak2014wikipedia}. Nel 2007, fu lanciato il VisualEditor, uno strumento che consente agli utenti di modificare le voci utilizzando un'interfaccia WYSIWYG (What You See Is What You Get), rendendo più semplice la partecipazione anche per coloro che non hanno familiarità con il markup wiki \cite{history_of_wikis}.

La comunità di Wikipedia ha continuato a crescere, con un numero sempre maggiore di utenti che contribuiscono al miglioramento e all'espansione delle voci. La partecipazione attiva della comunità è uno degli elementi chiave del successo di Wikipedia, poiché permette una revisione continua e collaborativa dei contenuti. Gli utenti possono discutere sulle modifiche da apportare alle voci nelle pagine di discussione, promuovendo un ambiente di collaborazione e miglioramento continuo \cite{denning2005wikipedia}.

\begin{figure}
    \includegraphics[width=\linewidth]{Immagini/Grafico pagine wikipedia excel.png}
    \caption{Numero complessivo di pagine di Wikipedia in lingua inglese e italiana in riferimento all'anno, dal 2001 ad oggi.\cite{wikimedia_stats} }
    \captionsetup{width=.8\linewidth}
    \label{fig:wikiPagesYear}
\end{figure}

Wikipedia è diventata disponibile in un numero crescente di lingue, riflettendo la sua missione di rendere la conoscenza accessibile a livello globale. Attualmente, Wikipedia esiste in oltre 300 lingue, con milioni di voci che coprono una vasta gamma di argomenti. Le edizioni in lingue diverse dall'inglese continuano a crescere, contribuendo a rendere Wikipedia una risorsa veramente globale \cite{reagle2010good}.

Nel corso degli anni, Wikipedia ha affrontato e superato numerose sfide, tra cui problemi di vandalismo, questioni di qualità delle informazioni e critiche sulla sua affidabilità. Per affrontare questi problemi, la comunità di Wikipedia ha sviluppato una serie di politiche e linee guida, come il controllo delle modifiche e l'adozione di strumenti automatizzati (bot) per monitorare e migliorare la qualità dei contenuti \cite{jemielniak2014wikipedia}.

Grazie a questi sforzi, Wikipedia è diventata una delle risorse più consultate al mondo, utilizzata quotidianamente da milioni di persone per ottenere informazioni su una vasta gamma di argomenti. La sua evoluzione e crescita nel tempo dimostrano il potere della collaborazione aperta e della condivisione della conoscenza.

\subsection{Milestones e sviluppi significativi}

Nel 2003, la creazione della Wikimedia Foundation rappresentò un importante passo avanti per Wikipedia. La fondazione, una organizzazione senza scopo di lucro, fu istituita per fornire supporto tecnico, legale e finanziario al progetto. Questo ha permesso a Wikipedia di crescere in modo sostenibile e di mantenere la sua indipendenza editoriale \cite{reagle2010good}.

Nel 2004, fu introdotto il sistema di categorie, che ha migliorato notevolmente l'organizzazione e la navigazione delle voci su Wikipedia. Questo sistema permette di raggruppare le voci in base a temi comuni, facilitando la ricerca delle informazioni e la scoperta di contenuti correlati \cite{jemielniak2014wikipedia}.

Un'altra importante milestone fu l'introduzione del VisualEditor nel 2013. Questo strumento ha rivoluzionato il modo in cui gli utenti possono modificare le voci, offrendo un'interfaccia WYSIWYG (What You See Is What You Get) che semplifica l'editing anche per coloro che non hanno familiarità con il markup wiki \cite{history_of_wikis}. Il VisualEditor ha reso la modifica delle pagine più accessibile, incoraggiando una partecipazione più ampia.

Nel corso degli anni, Wikipedia ha implementato vari strumenti di controllo della qualità e meccanismi di revisione. Tra questi, il sistema di "flagged revisions", introdotto nel 2008, che permette a determinati utenti di revisionare e approvare modifiche prima che queste diventino visibili al pubblico. Questo ha contribuito a migliorare la qualità e l'affidabilità delle informazioni presenti su Wikipedia \cite{denning2005wikipedia}.

Un'importante evoluzione nella gestione dei contenuti è stata l'integrazione con Wikidata nel 2012. Wikidata è un database collaborativo che fornisce dati strutturati per supportare Wikipedia e altri progetti della Wikimedia Foundation. Questa integrazione ha permesso di migliorare la coerenza e l'aggiornamento delle informazioni tra le varie versioni linguistiche di Wikipedia \cite{history_of_wikis}.

Wikipedia ha anche affrontato sfide legali e di censura in diversi paesi. Ad esempio, nel 2010, Wikipedia fu temporaneamente bloccata in Cina a causa di contenuti considerati sensibili dal governo cinese. La comunità di Wikipedia e la Wikimedia Foundation hanno lavorato costantemente per affrontare queste sfide, cercando di garantire l'accesso libero all'informazione per tutti \cite{jemielniak2014wikipedia}.

Nel 2021, Wikipedia ha celebrato il suo 20º anniversario, un traguardo significativo che ha evidenziato il suo ruolo cruciale nella diffusione della conoscenza libera. Con milioni di voci in centinaia di lingue, Wikipedia continua a crescere e a evolversi, mantenendo il suo impegno verso la neutralità, la verificabilità e il contenuto libero \cite{reagle2010good}.

\section{Impatto sulla divulgazione scientifica e culturale}

\subsection{Wikipedia come risorsa educativa}

Wikipedia è ampiamente riconosciuta come una risorsa educativa preziosa, utilizzata da studenti, insegnanti e ricercatori di tutto il mondo. La sua accessibilità gratuita e la vastità delle informazioni disponibili la rendono uno strumento ideale per l'apprendimento autodiretto e la ricerca preliminare \cite{reagle2010good}.

Gli studenti utilizzano frequentemente Wikipedia come punto di partenza per le loro ricerche. Le voci di Wikipedia forniscono panoramiche generali sugli argomenti, complete di riferimenti a fonti esterne che possono essere consultate per un approfondimento ulteriore. Questa caratteristica aiuta gli studenti a comprendere rapidamente i concetti di base e a identificare le risorse accademiche rilevanti \cite{denning2005wikipedia}.

Gli insegnanti trovano in Wikipedia uno strumento utile per integrare le loro lezioni. Possono utilizzare le voci di Wikipedia per assegnare letture agli studenti, sviluppare materiali didattici e stimolare discussioni in classe. Inoltre, Wikipedia offre una piattaforma per progetti educativi in cui gli studenti possono contribuire alla creazione e alla modifica delle voci, sviluppando così competenze di ricerca, scrittura e pensiero critico \cite{jemielniak2014wikipedia}.

Uno degli aspetti più significativi di Wikipedia come risorsa educativa è la sua capacità di promuovere l'alfabetizzazione digitale. Utilizzando Wikipedia, gli studenti imparano a valutare criticamente le informazioni, a distinguere tra fonti affidabili e non affidabili e a comprendere l'importanza della citazione delle fonti. Queste competenze sono essenziali in un'epoca in cui l'accesso alle informazioni è vasto e spesso caotico \cite{lih2009wikipedia}.

Wikipedia è anche uno strumento potente per la promozione dell'apprendimento continuo. Gli utenti di tutte le età possono utilizzare l'enciclopedia per esplorare nuovi interessi, approfondire conoscenze esistenti e rimanere aggiornati su una vasta gamma di argomenti. La natura collaborativa di Wikipedia permette agli utenti di condividere il loro sapere e di apprendere dagli altri, creando una comunità globale di apprendimento \cite{history_of_wikis}.

Le collaborazioni tra Wikipedia e istituzioni educative hanno portato a numerosi progetti di successo. Ad esempio, il programma Wikipedia Education Program coinvolge studenti e docenti in tutto il mondo, incoraggiandoli a contribuire a Wikipedia come parte del loro percorso accademico. Questi progetti non solo arricchiscono il contenuto di Wikipedia, ma forniscono anche agli studenti un'esperienza pratica nella ricerca e nella scrittura accademica \cite{jemielniak2014wikipedia}.

Nonostante le sue numerose qualità, Wikipedia affronta anche critiche come risorsa educativa. Alcuni educatori esprimono preoccupazioni riguardo alla qualità e all'affidabilità delle informazioni presenti su Wikipedia. Tuttavia, la comunità di Wikipedia e la Wikimedia Foundation lavorano costantemente per migliorare la qualità delle voci attraverso politiche di revisione, l'uso di fonti affidabili e l'integrazione di strumenti automatizzati per il controllo delle modifiche \cite{denning2005wikipedia}.

In conclusione, Wikipedia rappresenta una risorsa educativa versatile e potente, capace di supportare l'apprendimento in molti modi. La sua accessibilità, la vastità dei contenuti e la capacità di promuovere l'alfabetizzazione digitale la rendono uno strumento indispensabile nell'educazione moderna.

\subsection{La qualità e l'affidabilità delle informazioni}

La qualità e l'affidabilità delle informazioni su Wikipedia sono temi di fondamentale importanza, sia per la comunità degli utenti che per i lettori. Essendo un'enciclopedia aperta e collaborativa, Wikipedia deve affrontare la sfida di mantenere standard elevati di accuratezza e affidabilità nonostante la possibilità di modifiche da parte di chiunque \cite{reagle2010good}.

Un elemento chiave per garantire la qualità delle informazioni è il principio di verificabilità. Ogni informazione presente su Wikipedia deve essere supportata da fonti affidabili che possono essere controllate dai lettori. Questo principio richiede che gli editori citino fonti autorevoli, come libri accademici, articoli scientifici e pubblicazioni giornalistiche, per garantire che i contenuti siano basati su prove concrete e verificabili \cite{denning2005wikipedia}.

Wikipedia adotta una politica di neutralità, il che significa che le voci devono essere scritte in modo imparziale, presentando i fatti senza pregiudizi. Questo è essenziale per mantenere la credibilità dell'enciclopedia e per garantire che le informazioni fornite siano equilibrate e rappresentative dei diversi punti di vista su un argomento \cite{reagle2010good}.

Wikipedia utilizza anche bot, ovvero programmi automatizzati, per eseguire compiti ripetitivi come il controllo dei link interrotti, la rimozione di spam e l'aggiornamento delle informazioni. Questi bot contribuiscono a mantenere l'enciclopedia in ordine e a garantire che le informazioni siano sempre aggiornate e accurate \cite{history_of_wikis}.

Le politiche editoriali di Wikipedia richiedono che le voci siano scritte in un linguaggio chiaro e comprensibile, evitando l'uso di linguaggio tecnico non necessario. Questo aiuta a rendere le informazioni accessibili a un pubblico ampio e diversificato, migliorando la comprensione e l'usabilità delle voci \cite{reagle2010good}.

Nonostante queste misure, Wikipedia affronta critiche riguardo alla qualità e all'affidabilità delle informazioni. Alcuni studiosi e professionisti esprimono preoccupazioni sul fatto che l'apertura della piattaforma possa portare a errori e informazioni fuorvianti. Tuttavia, studi comparativi hanno dimostrato che, in molti casi, la qualità delle informazioni su Wikipedia è paragonabile a quella delle enciclopedie tradizionali, come l'Enciclopedia Britannica \cite{giles2005nature}.

Per affrontare queste critiche, la comunità di Wikipedia continua a sviluppare e migliorare le politiche di revisione e controllo della qualità. Questo include l'implementazione di strumenti avanzati per l'analisi delle modifiche, la formazione di gruppi di lavoro specializzati e la collaborazione con esperti in vari campi del sapere. Questi sforzi collettivi aiutano a mantenere Wikipedia come una risorsa affidabile e rispettabile per milioni di utenti in tutto il mondo \cite{jemielniak2014wikipedia}.

\subsection{Contributi alla scienza e alla conoscenza pubblica}

Wikipedia ha avuto un impatto significativo sulla scienza e sulla conoscenza pubblica, fungendo da ponte tra il pubblico generale e la comunità scientifica. Essendo una delle risorse online più consultate al mondo, Wikipedia rende le informazioni scientifiche accessibili a milioni di persone, promuovendo una maggiore comprensione di argomenti complessi \cite{reagle2010good}.

Uno dei principali contributi di Wikipedia è la democratizzazione dell'accesso alla conoscenza. Prima dell'avvento di Wikipedia, l'accesso a informazioni di qualità era spesso limitato a coloro che potevano permettersi costose enciclopedie o iscrizioni a riviste accademiche. Wikipedia ha eliminato queste barriere, rendendo la conoscenza disponibile gratuitamente a chiunque abbia una connessione Internet \cite{lih2009wikipedia}.

Wikipedia svolge un ruolo cruciale nella diffusione delle scoperte scientifiche. Gli articoli su Wikipedia vengono frequentemente aggiornati per riflettere le ultime ricerche e scoperte, rendendo possibile per il pubblico rimanere informato sugli sviluppi più recenti in vari campi della scienza. Questo è particolarmente evidente nelle voci riguardanti la medicina, la fisica, la biologia e le scienze ambientali \cite{jemielniak2014wikipedia}.

La piattaforma è utilizzata anche come strumento educativo nelle università e nelle scuole. Molti docenti incoraggiano gli studenti a utilizzare Wikipedia come punto di partenza per le loro ricerche, grazie alla sua capacità di fornire panoramiche generali e riferimenti a fonti più dettagliate. Inoltre, diversi programmi educativi coinvolgono gli studenti nella creazione e nella modifica delle voci di Wikipedia, fornendo loro un'esperienza pratica nella ricerca e nella scrittura accademica \cite{denning2005wikipedia}.

Un ulteriore contributo di Wikipedia alla qualità e all'affidabilità delle informazioni risiede nel suo sostegno alla scienza aperta (open science). La scienza aperta promuove la trasparenza, la condivisione dei dati e la collaborazione tra ricercatori di tutto il mondo. Wikipedia, con il suo modello aperto e collaborativo, incarna molti dei principi della scienza aperta, rendendo le informazioni scientifiche accessibili a chiunque. Le voci scientifiche di Wikipedia spesso includono riferimenti a ricerche e dati disponibili liberamente, incoraggiando la condivisione del sapere e il coinvolgimento della comunità scientifica e del pubblico generale. Questo approccio non solo aumenta la trasparenza, ma facilita anche la verifica delle informazioni e la replicazione degli studi, elementi fondamentali per il progresso scientifico \cite{fecher2014open}, \cite{nielsen2012reinventing}.

Wikipedia ha anche collaborato con numerose istituzioni scientifiche e accademiche per migliorare la qualità delle sue voci. Queste collaborazioni includono il Wikimedia Research Program, che coinvolge ricercatori nell'analisi e nel miglioramento dei contenuti di Wikipedia, e il progetto GLAM (Galleries, Libraries, Archives, and Museums), che promuove la collaborazione tra Wikipedia e istituzioni culturali per condividere la conoscenza e le risorse \cite{jemielniak2014wikipedia}.

Nonostante i suoi contributi, Wikipedia affronta anche sfide significative. La qualità delle informazioni può variare e ci sono preoccupazioni riguardo alla rappresentazione equa delle diverse discipline scientifiche. Tuttavia, la comunità di Wikipedia lavora costantemente per migliorare la qualità e la copertura delle voci, attraverso revisioni continue e politiche di verifica rigorose \cite{denning2005wikipedia}.

In conclusione, Wikipedia rappresenta una risorsa inestimabile per la scienza e la conoscenza pubblica. La sua capacità di rendere le informazioni scientifiche accessibili a un vasto pubblico, insieme al suo impegno per la qualità e la trasparenza, la rende uno strumento cruciale nel panorama della conoscenza globale.

\subsection{Critiche e sfide affrontate da Wikipedia}

Fin dalla sua fondazione, Wikipedia ha dovuto affrontare numerose critiche e sfide che riguardano vari aspetti del suo funzionamento. Una delle critiche più frequenti riguarda i conflitti editoriali. La comunità di Wikipedia è composta da una vasta gamma di utenti con opinioni diverse, il che può portare a scontri su come presentare determinati argomenti. Questi conflitti possono rallentare il processo di modifica e talvolta portare a "guerre di modifica", dove gli utenti annullano ripetutamente le modifiche degli altri. Per mitigare questi problemi, Wikipedia ha sviluppato linee guida e politiche per la risoluzione delle dispute e ha istituito il ruolo degli amministratori per facilitare la mediazione \cite{reagle2010good}.

Un'altra sfida significativa è la rappresentazione e il bias. Alcuni studi hanno evidenziato che Wikipedia può riflettere pregiudizi culturali, di genere e geografici, a causa della predominanza di contributori provenienti da determinate regioni e background. Ad esempio, uno studio del 2011 ha rilevato che gli articoli di Wikipedia tendono a rappresentare prevalentemente prospettive occidentali e maschili, creando uno squilibrio nella copertura degli argomenti \cite{lam2011wp}. La Wikimedia Foundation e la comunità di Wikipedia stanno lavorando per affrontare questi problemi attraverso iniziative volte a diversificare la base dei contributori e a migliorare la rappresentazione dei contenuti.

La sostenibilità finanziaria è un'altra sfida cruciale per Wikipedia. La piattaforma è gestita dalla Wikimedia Foundation, che si basa principalmente su donazioni per finanziare le operazioni. Sebbene Wikipedia riceva donazioni significative da tutto il mondo, la dipendenza dalle donazioni rende la sostenibilità a lungo termine una questione aperta. La Wikimedia Foundation ha esplorato diverse strategie per diversificare le fonti di finanziamento, mantenendo al contempo l'accesso gratuito e libero alla piattaforma \cite{history_of_wikis}.

La protezione della privacy e la sicurezza degli utenti sono temi rilevanti. Wikipedia permette modifiche anonime, ma ciò ha sollevato preoccupazioni riguardo alla responsabilità e alla tracciabilità delle modifiche. Inoltre, i contributori possono essere esposti a rischi di privacy, specialmente in contesti politici sensibili. La Wikimedia Foundation ha implementato politiche di privacy e strumenti per proteggere gli utenti, ma queste misure devono essere continuamente aggiornate per affrontare nuove minacce \cite{denning2005wikipedia}.

Nonostante queste sfide, studi comparativi hanno dimostrato che la qualità delle informazioni su Wikipedia è spesso paragonabile a quella di altre enciclopedie tradizionali. Ad esempio, un articolo pubblicato su "Nature" nel 2005 ha confrontato Wikipedia con l'Enciclopedia Britannica e ha trovato che la qualità degli articoli scientifici era sorprendentemente simile tra le due fonti \cite{giles2005nature}. Questo risultato ha aiutato a migliorare la percezione pubblica di Wikipedia come fonte affidabile di informazioni.

In conclusione, Wikipedia continua a essere una risorsa fondamentale per milioni di persone in tutto il mondo. La comunità di Wikipedia è resiliente e continua a lavorare per migliorare la piattaforma, affrontando le critiche e le sfide attraverso l'innovazione e la collaborazione. Le iniziative per migliorare la qualità, diversificare la comunità dei contributori e garantire la sostenibilità finanziaria sono in corso e rappresentano passi cruciali per il futuro di Wikipedia \cite{reagle2010good}.

\chapter{Introduzione alla Spettroscopia}

La spettroscopia come branca dell'ottica si fa storicamente iniziare nel XVII secolo con Isaac Newton, il primo ad utilizzare il termine "spettro", per descrivere la dispersione della luce bianca nei colori che la compongono per effetto di un prisma\cite{Newton1671}.

Newton raccolse i sui esperimenti sul fenomento in un trattato di tre volumi, Opticks\cite{newton1704opticks}, che pose le basi per l'ottica moderna, confutando la teoria del colore allora accettata, di origine aristotelica, la quale attribuiva i colori al grado di diafanità della luce\cite{boscarol2024}, e proponendo il modello corpuscolare della luce.

Nel XIX secolo, Joseph von Fraunhofer perfezionò diversi strumenti ottici, tra cui prismi e reticoli di diffrazione, e li utilizzò per osservare lo spettro solare e identificarne le linee scure, ora note come linee di Fraunhofer\cite{Fraunhofer1817}. A Fraunhofer si deve l'introduzione dell'analisi quantitativa alla spettroscopia, in quanto fu il primo in grado di misurare con precisione le lunghezze d'onda della luce.

Huygens propose che la luce si propagasse come un'onda e introdusse il principio che porta il suo nome\cite{Huygens1690}. Secondo il principio di Huygens, ogni punto di un fronte d'onda può essere considerato come una sorgente di onde secondarie sferiche, che si propagano in tutte le direzioni; la somma di queste onde secondarie determina la posizione del fronte d'onda successivo. Questo principio poteva spiegare fenomeni come la rifrazione e la diffrazione della luce.

Successivamente, Thomas Young verificò la natura ondulatoria della luce per mezzo del suo famoso esperimento di interferenza\cite{Young1804}.

La spettroscopia si sviluppò ulteriormente con Gustav Kirchhoff e Robert Bunsen, che collegarono univocamente le linee di emissione e assorbimento agli elementi chimici dai quali erano prodotte\cite{Kirchhoff1861}, stabilendo la base per l'analisi chimica mediante la spettroscopia. La spettroscopia divenne così uno strumento essenziale in chimica, fisica e astronomia, permettendo di identificare la composizione chimica delle stelle e di altri corpi celesti.

Nel XX secolo, Niels Bohr propose un modello atomico quantizzato che poteva sipegare la regolarità delle linee spettrali degli atomi idrogenoidi\cite{Bohr1913}.

Le tecniche spettroscopiche si sono evolute con l'introduzione della spettroscopia Raman e della spettroscopia infrarossa, che sfruttano la diffusione inelastica e le vibrazioni molecolari per analizzare la composizione chimica dei materiali. La spettroscopia laser, sviluppata negli anni '60, ha permesso misurazioni estremamente precise delle frequenze luminose, rivoluzionando l'analisi spettroscopica.

\section{Rifrazione e Legge di Snell}

La rifrazione è un fenomeno ottico che si verifica quando un'onda luminosa passa da un mezzo a un altro con una diversa densità ottica, o indice di rifrazione, provocando un cambiamento nella velocità dell'onda e, conseguentemente, nella direzione di propagazione dell'onda stessa. Quando un onda colpisce obliquamente un mezzo con indice di rifrazione diverso, ad esempio maggiore, subirà una variazione di velocità in maniera non uniforme, in quanto la prima parte dell'onda che entra in contatto con il nuovo mezzo verrà rallentata per prima. Questo causa una deformazione del fronte d'onda, e conseguentemente un cambio di direzione, come rappresentato in \cref{fig:wawefrontRefraction}.

\begin{figure}[!ht]
    \centering
    \includegraphics[width=150px]{Immagini/Snells_law_wavefronts.png}
    \captionsetup{width=.8\linewidth}
    \caption[\textit{By Oleg Alexandrov}]{\textit{By Oleg Alexandrov\footnotemark} Rappresentazione della deformazione subita da un'onda sferica al passaggio da un mezzo a indice di rifrazione minore ad uno con indice di rifrazione maggiore. La linea rossa, normale al fronte d'onda, evidenzia la rifrazione}
    \label{fig:wawefrontRefraction}
\end{figure}
\footnotetext{By Oleg Alexandrov, Public Domain, \href{https://commons.wikimedia.org/w/index.php?curid=3323373}{https://commons.wikimedia.org/w/index.php?curid=3323373}}

Il rallentamento della luce in in mezzo a maggiore densità non è dovuto a fenomeni di scattering o assorbimento, ma all'eccitazione degli elettroni del mezzo stesso che agiscono come un induttore. Questi infatti, stimolati dalla carica elettrica dei fotoni, iniziano a loro volta ad oscillare, producendo onde elettromagnetiche che fanno interferenza con l'onda di onde luminosa, producendo un'onda con frequenza minore\cite{Fermilab2019}.

\subsection{Derivazione della Legge di Snell dal Principio di Fermat}

La legge di Snell può essere derivata utilizzando il principio di Fermat, che afferma che il percorso seguito dalla luce tra due punti è quello che richiede il tempo minimo.

Consideriamo un raggio di luce, perpendicolare al fronte d'onda, che passa da un mezzo con indice di rifrazione \( n_1 \) a un altro mezzo con indice di rifrazione \( n_2 \) come in \cref{fig:snellsLaw}. Supponiamo che il raggio inizi nel punto \( Q \) nel primo mezzo e termini nel punto \( P \) nel secondo mezzo.

\begin{figure}[!ht]
    \centering
    \includegraphics[width=\linewidth]{Immagini/Snells_law_Diagram_B_vector.png}
    \captionsetup{width=.8\linewidth}
    \caption[\textit{By Oleg Alexandrov}]{\textit{By Smedlib\footnotemark} Rappresentazione di un raggio luminoso al passaggio da un mezzo con indice di rifrazione minore \( n_1 \) ad uno con indice di rifrazione maggiore \( n_2 \)}
    \label{fig:snellsLaw}
\end{figure}
\footnotetext{By Smedlib, CC BY-SA 4.0, \href{https://commons.wikimedia.org/w/index.php?curid=60639100}{https://commons.wikimedia.org/w/index.php?curid=60639100}}


Il tempo impiegato dalla luce per percorrere il tratto \( Q \to O \to P \), dove \( O \) è il punto di incidenza sulla superficie di separazione, è dato da:

\[ t = \frac{QO}{v_1} + \frac{OP}{v_2} \]

dove \( v_1 \) e \( v_2 \) sono le velocità della luce nei rispettivi mezzi. Poiché la velocità della luce in un mezzo è inversamente proporzionale all'indice di rifrazione \( v_i = \frac{c}{n_i} \), possiamo riscrivere il tempo totale come:

\[ t = \frac{\sqrt{x^2 + a^2}}{v_1} + \frac{\sqrt{(l - x)^2 + b^2}}{v_2} \]
\[ t = \frac{\sqrt{x^2 + a^2} \, n_1}{c} + \frac{\sqrt{(l - x)^2 + b^2} \, n_2}{c} \]

Per trovare il tempo minimo deriviamo rispetto a \( x \):

\[ \frac{d}{dx} \left( \frac{\sqrt{x^2 + a^2} \, n_1}{c} + \frac{\sqrt{(l - x)^2 + b^2} \, n_2}{c} \right) = 0 \]

\[ \frac{n_1 x}{c \sqrt{x^2 + a^2}} - \frac{n_2 (l - x)}{c \sqrt{(l - x)^2 + b^2}} = 0 \]

Semblificando  \( c \) otteniamo:

\[ \frac{n_1 x}{\sqrt{x^2 + a^2}} = \frac{n_2 (l - x)}{\sqrt{(l - x)^2 + b^2}} \]

Riconoscendo che il seno degli angoli di incidenza (\( \theta_1 \)) e di rifrazione (\( \theta_2 \)) è rispettivamente dato da:

\[ \sin \theta_1 = \frac{x}{\sqrt{x^2 + a^2}} \]
\[ \sin \theta_2 = \frac{l - x}{\sqrt{(l - x)^2 + b^2}} \]

Possiamo scrivere:

\[ n_1 \sin \theta_1 = n_2 \sin \theta_2 \]

\section{Interferenza: L'Esperimento di Young}

\subsection{Descrizione dell'esperimento di Young}

L'esperimento di interferenza di Thomas Young, condotto nel 1801, è uno dei più celebri nella storia della fisica. Questo esperimento dimostrò la natura ondulatoria della luce e fu cruciale per la comprensione del fenomeno dell'interferenza. Nell'esperimento, Young fece entrare la luce solare in una stanza buia attraverso un foglio di carta spessa appositamente forato con un ago, in modo da ottenere un fascio luminoso orizzontale e sufficientemente collimato. Il raggio così ottenuto andava a proiettarsi sulla parete opposta a quella su cui si trovava la finestra da cui era entrato, sulla quale era possibile riscontrare gli effetti dell'interferenza.

Lungo il percorso del fascio di luce, ad una distanza nota dalla parete, era posto un ostacolo in grado di dividere il raggio in due. Nonostante sia passato alla storia come esperimento della doppia fenditura, nella prima versione Young aveva utilizzato una carta dello spessore di qualche decimo di millimetro. Il raggio, così diviso, andava a formare sulla parete una figura d'interferenza, che scompariva coprendone una delle due metà. Questo dimostrava che la figura di interferenza fosse dovuta all'interazione tra le due metà del fascio luminoso e non provocata dall'ostacolo.

La semplicità di questo esperimento e la sua rilevanza storica lo rendono ideale per la riproduzione da parte degli studenti, anche in mancanza di laboratori universitari. Sostituendo la luce solare con quella di un comune laser tascabile infatti, l'esperimento di Young è stato proposto nelle classi di liceo\cite{Scheider1986}. Con le restrizioni dovute alla pandemia scatenata dal SARS-CoV-2, che hanno imposto lo svolgimento delle lezioni in remoto, l'esperimento di Young è stato anche proposto agli studenti dell'università di Bologna come attività di laboratorio da svolgere direttamente a casa propria, utilizzando un CD come reticolo di diffrazione\cite{Campari2021}. 

\begin{figure}[!ht]
    \centering
    \includegraphics[width=\linewidth]{Immagini/double_slit_above.png}
    \captionsetup{width=.8\linewidth}
    \caption{Raffigurazione dell'esperimento della doppia fenditura.}
    \label{fig:doubleSlitAbove}
\end{figure}

In \cref{fig:doubleSlitAbove} è raffigurata la versione dell'esperimento che utilizza una doppia fenditura per separare il fascio luminoso. La trattazione numerica è identica, in quanto l'effetto di intereferenza è dato dalla differenza di cammino ottico \( \Delta r \) tra i due raggi e non dall'interazione con l'ostacolo, ma la rappresentazione risulta più chiara. Perciò sarà utilizzata questa versione per ricavare la posizione dei massimi e dei minimi di interferenza.

\subsection{Posizione dei massimi e dei minimi di interferenza}

Per trovare la posizione delle bande luminose, i massimi di interferenza, e le zone buie, i minimi di interfereza, nell'esperimento della doppia fenditura, consideriamo un'onda luminosa coerente e monocromatica, proveniente ad esempio da un laser, la quale incide su una barriera con due fenditure separate da una distanza \(d\). 

La luce che passa attraverso le fenditure crea un pattern di interferenza su uno schermo situato a una distanza \(L\) dalle fenditure, con \(L \gg d \), in modo tale che i raggi \(r_1\) ed \(r_2\), che partono dalle due fessure e finiscono nello stesso punto sulla parete, possano essere considerati come paralleli.

\begin{figure}[!ht]
    \centering
    \includegraphics[width=\linewidth]{Immagini/double_slit_setup.png}
    \captionsetup{width=.8\linewidth}
    \caption{raffigurazione dell'esperimento della doppia fenditura visto dall'alto. Il fascio luminoso coerente originato da \d viene diviso in due }
    \label{fig:doubleSlitSetup}
\end{figure}

Da \cref{fig:doubleSlitSetup} è anche facile vedere che l'angolo \(\theta\) di inclinazione dei raggi uscenti dalle fenditure è uguale all'angolo acuto del triangolo rettangolo che ha come ipotenusa \(O_1, O_2 \).

Questo comporta che la differenza di cammino ottico tra le due onde sia data da:

\[ \Delta r = d \sin \theta \]

dove \(\theta\) è l'angolo di deviazione della luce rispetto alla perpendicolare alle fenditure.

I massimi di interferenza si verificano quando la differenza di cammino ottico è un multiplo intero della lunghezza d'onda \(\lambda\):
 
\[ d \sin \theta = m \lambda \]

dove \(m\) (\(m = 0, \pm 1, \pm 2, \ldots\)) viene detto "ordine" del massimo di interferenza.
  
I minimi di interferenza si verificano quando la differenza di cammino ottico è un multiplo, dispari, di metà lunghezza d'onda: 

\[ d \sin \theta = \left( m + \frac{1}{2} \right) \lambda \]

Per \(\theta\) piccoli, possiamo approssimare \(\sin \theta \approx \tan \theta\). Se \(y\) è la distanza verticale dalla frangia centrale (frangia di ordine \(m=0\)) sullo schermo, allora:

\[ \tan \theta \approx \sin \theta \approx \frac{y}{L} \]

Sostituendo nella condizione per i massimi di interferenza:

\[ d \frac{y}{L} = m \lambda \]

Da cui la posizione delle frange di ordine \(m\) è:

\[ y = \frac{m \lambda L}{d} \]

Sostituendo nella condizione per i minimi di interferenza:

\[ d \frac{y}{L} = \left( m + \frac{1}{2} \right) \lambda \]

Da cui la posizione delle ombre di ordine \(m\) è:

\[ y = \frac{\left(m + \frac{1}{2} \right) \lambda L}{d} \]

\section{Emissione e Assorbimento: Il Modello Atomico di Bohr}

Il modello atomico di Bohr, introdotto da Niels Bohr nel 1913, rappresenta un fondamentale cambio di paradigma nella comprensione della struttura atomica. Prima del modello di Bohr, il modello planetario di Rutherford descriveva l'atomo come un nucleo centrale attorno al quale orbitavano gli elettroni, similmente ai pianeti attorno al Sole. Tuttavia, questo modello presentava un problema significativo: secondo l'elettrodinamica classica, gli elettroni in movimento avrebbero dovuto emettere radiazione elettromagnetica, perdendo energia e spiraleggiando verso il nucleo, portando all'instabilità dell'atomo. Bohr risolse questo problema proponendo che gli elettroni potessero occupare solo orbite discrete e stabili, in cui non emettono radiazione \cite{Bohr1913}.

Bohr propose tre postulati principali per risolvere le limitazioni del modello di Rutherford:

\begin{enumerate}
    \item \textbf{Orbite Stazionarie:} Gli elettroni possono orbitare attorno al nucleo solo in certe orbite discrete senza emettere radiazione. Queste orbite stazionarie corrispondono a specifici livelli di energia quantizzata \cite{BohrModelBritannica}.
    \item \textbf{Quantizzazione del Momento Angolare:} Il momento angolare degli elettroni in orbita è quantizzato e può assumere solo valori che sono multipli interi della costante di Planck ridotta (\(\hbar = h/2\pi\)):
        \[L = n\hbar = n\frac{h}{2\pi}\]
        dove \( n \) è un numero intero positivo (noto come numero quantico principale) \cite{HyperPhysicsBohr}.
 
    \item \textbf{Transizioni Elettroniche e Emissione/Assorbimento di Energia:} Gli elettroni possono cambiare orbita solo assorbendo o emettendo un fotone con energia pari alla differenza tra i livelli energetici:
        \[\Delta E = E_2 - E_1 = h\nu\]
        dove \( h \) è la costante di Planck e \( \nu \) è la frequenza della radiazione emessa o assorbita \cite{Bohr1913}.
 \end{enumerate}

\subsection{Raggio di Bohr}

Il modello atomico di Bohr risulta particolarmente efficace per ricavare i livelli energetici degli atomi idrogenoidi, e di conseguenza è capace di predirne con buona precisione gli spettri di emissione ed assobimento.

Se \(m_e\) è la massa dell'elettrone, \(v\) la velocità ed \(r\) il raggio dell'orbita, allora il suo momento angolare quantizzato sarà dato da:

\[ L = m_e v r = n \hbar \]

Da cui

\[ v = \frac{n \hbar}{m_e r} \]

Perchè l'orbita sia stabile l'attrazione coulumbiana dovrà essere bilanciata dalla forza centrifuga:

\[ \frac{m_e v^2}{r} = \frac{Ze^2}{4 \pi \epsilon_0 r^2} \]

dove \(Z\) è il numero di protoni nel nucleo, \(e\) è la carica dell'elettrone ed \(\epsilon\) la costante dielettrica del vuoto. Sostituendo \(v\):

\[ \frac{m_e \left(\frac{n \hbar}{m_e r}\right)^2}{r} = \frac{Ze^2}{4 \pi \epsilon_0 r^2} \]

Semplificando e risolvendo per \(r\):

\[ r = \frac{4 \pi \epsilon_0 n^2 \hbar^2}{Zm_e e^2} \]

Che per l'orbita più piccola \(n = 1\) l'atomo di idrogeno \(Z = 1\) da il raggio di Bohr:

\[ a_0 = \frac{4 \pi \epsilon_0 \hbar^2}{m_e e^2} \approx 0,53 \AA \]

\subsection{Livelli Energetici}

L'energia totale \(E\) dell'elettrone è data dalla somma della sua energia cinetica \(K\) e di quella potenziale coulumbiana \(U\):

\[ E = K + U = \frac{1}{2} m_e v^2 - \frac{Ze^2}{4 \pi \epsilon_0 r} \]

Sostituendo \(v\):

\[ E = \frac{1}{2} m_e \left(\frac{n \hbar}{m_e r}\right)^2 - \frac{Ze^2}{4 \pi \epsilon_0 r} \]

E successivamente \(r\):

\[ E = - \frac{m_e Z e^4}{8 \epsilon_0^2 h^2} \frac{1}{n^2} \]

\subsection{Costante di Rydberg}

La legge di Rydberg è una legge empirca formulata da Johannes Rydberg nel 1888, che descrive le lunghezze d'onda degli spettri di emissione e assorbimento degli atomi idrogenoidi. Se \(\lambda\) è la lunghezza d'onda della luce emessa o assorbita allora secondo la legge di Rydberg:

\[ \frac{1}{\lambda} = R_H \left( \frac{1}{n_1^2} - \frac{1}{n_2^2} \right) \]

dove \(R_H\) è la costante di Rydberg, e \(n_1\) e \(n_2\) sono numeri interi.

Considerando la quantizzazione dell'energia nell'atomo di Bohr avremo:
\[\Delta E = E_2 - E_1 = h\nu = \frac{hc}{\lambda}\]

Recuperando i livelli energetici degli atomi idrogenoidi e risolvendo per \(1/\lambda\):

\[\frac{1}{\lambda}=\frac{1}{hc}\left(E_2 - E_1\right) 
    = \frac{1}{hc} \frac{m_e Z e^4}{2 \epsilon_0^2 \hbar^2}\left(\frac{1}{n_1^2} - \frac{1}{n_2^2}\right)
    = \frac{R_E}{hc}\left(\frac{1}{n_1^2} - \frac{1}{n_2^2}\right) = R_H\left(\frac{1}{n_1^2} - \frac{1}{n_2^2}\right) \]

\[ R_H = \frac{m_e e^4}{8 \epsilon_0^2 h^2} \]

Dove \(R_E\) prendei l nome di energia di Rydberg,

\subsection{Dopo l'atomo di Bohr}

La capacità del modello di Bohr di spiegare le linee spettrali discrete di emissione ed assorbimento degli atomi idrogenoidi e di dare supporto teorico alla legge empirica di Rydberg ne garantirono l'adozione.

Tuttavia, il modello, presentava alcune limitazioni. Non poteva spiegare gli spettri di atomi più complessi rispetto agli idrogenoidi e non teneva conto degli effetti di schermatura dovuti alle interazioni tra più elettroni. Inoltre, non poteva giustificare completamente alcune osservazioni sperimentali, come le doppie linee osservate in alcuni spettri, dovute all'effetto Zeeman \cite{ZeemanEffect}.

Dopo il modello di Bohr, la comprensione della struttura atomica si è evoluta con lo sviluppo della meccanica quantistica. Erwin Schrödinger introdusse l'equazione di Schrödinger nel 1926, descrivendo gli elettroni come onde stazionarie in orbite probabilistiche.

\chapter{Redazione della pagina}
\section{Come scrivere su Wikipedia}

Scrivere su Wikipedia richiede attenzione, rispetto delle linee guida e un approccio collaborativo. In questa sezione, verranno illustrate le linee guida per la redazione delle voci, le politiche di verifica e neutralità, e gli strumenti e le risorse disponibili per i contributori.

\subsection{Linee guida per la redazione delle voci}

Le linee guida per la redazione delle voci su Wikipedia sono fondamentali per garantire la qualità e la coerenza dei contenuti. Alcuni dei principi chiave includono:

\begin{itemize}
    \item \textbf{Neutralità}: Le voci devono essere scritte da un punto di vista neutrale, presentando i fatti in modo imparziale senza promuovere un'opinione particolare.
    \item \textbf{Verificabilità}: Ogni informazione deve essere supportata da fonti attendibili e verificabili. È essenziale citare fonti autorevoli come libri accademici, articoli scientifici e pubblicazioni giornalistiche.
    \item \textbf{No alla ricerca originale}: Wikipedia non è il luogo per presentare nuove teorie o ricerche originali. Tutte le informazioni devono essere basate su conoscenze già pubblicate e accettate.
    \item \textbf{Chiarezza e concisione}: Le voci devono essere scritte in modo chiaro e comprensibile, evitando linguaggi tecnici non necessari. È importante essere concisi e mantenere il focus sull'argomento trattato.
\end{itemize}

\subsection{Politiche di verifica e neutralità}

Le politiche di verifica e neutralità sono fondamentali per mantenere l'affidabilità e la credibilità di Wikipedia. Queste politiche includono:

\begin{itemize}
    \item \textbf{Punto di Vista Neutrale (NPOV)}: Questa politica richiede che tutte le voci siano scritte in modo imparziale, rappresentando tutti i punti di vista rilevanti in modo equo. I redattori devono evitare di esprimere opinioni personali e basarsi su fonti affidabili.
    \item \textbf{Verificabilità}: Ogni informazione presente su Wikipedia deve poter essere verificata da fonti esterne affidabili. Gli editori devono fornire riferimenti dettagliati che permettano ai lettori di confermare la validità delle informazioni.
    \item \textbf{Cite le fonti}: È essenziale citare fonti attendibili per ogni affermazione significativa. Le fonti devono essere di alta qualità e appropriate per l'argomento trattato.
    \item \textbf{Evitare conflitti di interesse}: Gli editori devono evitare di scrivere su argomenti nei quali hanno un interesse personale diretto, poiché ciò può compromettere la neutralità della voce.
\end{itemize}

\subsection{Strumenti e risorse per i contributori}

Wikipedia mette a disposizione una serie di strumenti e risorse per aiutare i contributori a migliorare le voci e a rispettare le linee guida. Alcuni degli strumenti più utili includono:

\begin{itemize}
    \item \textbf{Sandbox}: Uno spazio personale dove i contributori possono sperimentare e modificare le voci prima di pubblicarle. È un luogo ideale per fare prove senza influenzare le voci pubbliche.
    \item \textbf{VisualEditor}: Un editor visuale che permette di modificare le voci senza conoscere il markup wiki. È particolarmente utile per i nuovi utenti.
    \item \textbf{Pagine di discussione}: Ogni voce ha una pagina di discussione dove gli utenti possono confrontarsi e discutere su come migliorare i contenuti.
    \item \textbf{Guida di stile}: Un insieme di linee guida che aiuta a mantenere la coerenza stilistica e formattazione delle voci.
    \item \textbf{Progetti tematici}: Gruppi di utenti che collaborano su specifici argomenti, fornendo supporto e risorse per migliorare le voci in quel campo.
    \item \textbf{Bot}: Programmi automatizzati che eseguono compiti ripetitivi, come il controllo dei link rotti, l'aggiornamento delle categorie e la correzione di errori comuni.
\end{itemize}

\section{Analisi delle voci esistenti}

In questa sezione verrà analizzato lo stato attuale delle voci esistenti su Wikipedia riguardanti la spettroscopia, con particolare attenzione alla storia, alle principali tecniche e alle applicazioni della spettroscopia. Verranno evidenziate le lacune presenti nelle voci italiane e le opportunità di miglioramento attraverso la traduzione e l'integrazione di contenuti da altre lingue.

\subsection{Storia della spettroscopia}

La pagina di Wikipedia italiana dedicata alla spettroscopia risulta piuttosto scarna e non copre adeguatamente gli sviluppi storici di questa disciplina. In contrasto, la pagina inglese sulla storia della spettroscopia (\textit{History of spectroscopy}) offre una panoramica più dettagliata. Di seguito, viene riportata una traduzione dei punti salienti presenti nella voce inglese.

\subsubsection{Origini e primi sviluppi}

La spettroscopia come branca dell'ottica iniziò nel XVII secolo con Isaac Newton, il primo a utilizzare il termine "spettro" per descrivere la dispersione della luce bianca nei colori che la compongono per effetto di un prisma. Newton raccolse i suoi esperimenti sul fenomeno in un trattato di tre volumi, \textit{Opticks}, che pose le basi per l'ottica moderna.

Nel XIX secolo, Joseph von Fraunhofer perfezionò diversi strumenti ottici, tra cui prismi e reticoli di diffrazione, e li utilizzò per osservare lo spettro solare e identificarne le linee scure, ora note come linee di Fraunhofer. Fraunhofer fu il primo a misurare con precisione le lunghezze d'onda della luce, introducendo l'analisi quantitativa alla spettroscopia.

Successivamente, Thomas Young verificò la natura ondulatoria della luce attraverso il suo famoso esperimento di interferenza, mentre Gustav Kirchhoff e Robert Bunsen collegarono univocamente le linee di emissione e assorbimento agli elementi chimici dai quali erano prodotte, stabilendo la base per l'analisi chimica mediante la spettroscopia.

Nel XX secolo, Niels Bohr propose un modello atomico quantizzato che poteva spiegare la regolarità delle linee spettrali degli atomi idrogenoidi, segnando un fondamentale cambio di paradigma nella comprensione della struttura atomica.

\subsection{Principali tecniche spettroscopiche}

La pagina italiana di Wikipedia non offre una trattazione dettagliata delle principali tecniche spettroscopiche. La pagina inglese su \textit{Spectroscopy} fornisce invece una descrizione approfondita delle varie tecniche utilizzate in spettroscopia. Ecco un riassunto delle tecniche più rilevanti:

\subsubsection{Spettroscopia di assorbimento}

Questa tecnica misura la quantità di luce assorbita da una sostanza in funzione della lunghezza d'onda. È utilizzata per identificare e quantificare diversi componenti chimici in una soluzione.

\subsubsection{Spettroscopia di emissione}

In questa tecnica, la sostanza viene eccitata con energia, e la luce emessa durante il ritorno allo stato fondamentale viene analizzata. È utilizzata per identificare elementi chimici in un campione.

\subsubsection{Spettroscopia Raman}

Basata sulla diffusione inelastica della luce, la spettroscopia Raman fornisce informazioni sulle vibrazioni molecolari, permettendo di identificare le sostanze e studiare le loro proprietà chimiche e fisiche.

\subsubsection{Spettroscopia infrarossa (IR)}

Utilizza la radiazione infrarossa per studiare le vibrazioni molecolari. È ampiamente utilizzata per identificare composti organici e inorganici.

\subsubsection{Spettroscopia NMR (Risonanza Magnetica Nucleare)}

Analizza il comportamento dei nuclei atomici in un campo magnetico. È una tecnica fondamentale per la determinazione della struttura delle molecole organiche.

\subsection{Applicazioni della spettroscopia}

Le applicazioni della spettroscopia sono molteplici e spaziano in vari campi della scienza e della tecnologia. La pagina inglese su \textit{Spectroscopy} offre una panoramica completa di queste applicazioni. Ecco un estratto delle principali applicazioni:

\subsubsection{Analisi chimica}

La spettroscopia è utilizzata per identificare e quantificare composti chimici in campioni. Tecniche come la spettroscopia di massa e la spettroscopia atomica sono essenziali per l'analisi chimica.

\subsubsection{Astrofisica}

Gli astronomi utilizzano la spettroscopia per studiare la composizione e le proprietà fisiche delle stelle, delle galassie e di altri corpi celesti. Permette di determinare la velocità, la temperatura e la composizione chimica degli oggetti astronomici.

\subsubsection{Medicina}

In campo medico, la spettroscopia è utilizzata per diagnosticare malattie e monitorare lo stato di salute dei pazienti. La spettroscopia NMR è fondamentale nella risonanza magnetica, mentre la spettroscopia infrarossa è utilizzata per l'analisi dei tessuti.

\subsubsection{Fisica}

La spettroscopia è fondamentale per lo studio delle proprietà dei materiali e delle interazioni tra luce e materia. È utilizzata per investigare le bande proibite nei semiconduttori, le transizioni energetiche nei gas e i processi di scattering nei solidi.

\subsubsection{Industria}

In ambito industriale, la spettroscopia è utilizzata per il controllo della qualità, l'analisi dei materiali e il monitoraggio dei processi di produzione. È impiegata nella produzione di semiconduttori, nella chimica dei polimeri e nella sintesi di nuovi materiali.

\subsubsection{Ambiente}

La spettroscopia è utilizzata per monitorare la qualità dell'aria e dell'acqua, rilevando la presenza di inquinanti e altre sostanze chimiche nocive. Tecniche come la spettroscopia UV-Vis e la spettroscopia a fluorescenza sono essenziali per l'analisi ambientale.


L'analisi delle voci esistenti su Wikipedia evidenzia la necessità di integrare e migliorare i contenuti della voce italiana riguardanti la spettroscopia, soprattutto per quanto riguarda la trattazione della fisica dei fenomeni coinvolti. Tradurre e arricchire le voci esistenti con informazioni più dettagliate e specifiche contribuirà a rendere Wikipedia una risorsa più completa e affidabile per gli utenti italiani.

\section{Integrazioni}

In questa sezione verranno proposte delle aggiunte alla pagina di Wikipedia sulla spettroscopia, con particolare attenzione ai contenuti riguardanti la fisica dei fenomeni coinvolti e alcune curiosità legate ai fenomeni atmosferici. Le motivazioni e gli obiettivi di queste integrazioni verranno discussi, insieme alla metodologia proposta per l'aggiunta dei nuovi contenuti.

\subsection{Proposta di nuovi contenuti}

Propongo di aggiungere due nuove sezioni alla pagina di Wikipedia sulla spettroscopia:

\subsubsection{Fisica}

Questa sezione tratterà i principali fenomeni fisici fondamentali per comprendere la spettroscopia. Ho scelto tre argomenti, rifrazione, diffrazione e interferenze e atomo di Bohr, che oltre a essere centrali nella storia della spettroscopia segnano anche un cambio di paradigma nella fisica in generale.

\paragraph{Rifrazione} Con lo studio della rifrazione si gettano le basi ad inizio 1700 per la spettroscopia e l'ottica moderna. La pubblicazione di Opticks di Newton contribuisce anche a cementare l'esperimento e l'osservazione come origine del pensiero deduttivo nella scienza, piuttosto che il dogma aristotelico.
\paragraph{Diffrazione e interferenza} Entrambi fenomeni fondamentali per la spettroscopia, in quanto la loro comprensione permette lo sviluppo di strumentazione estremamente avanzata. L'esperimento di Young, che ha provato l'esistenza dell'effetto di interferenza tra due raggi luminosi, ha anche contribuito all'accettazione della natura ondulatoria della luce.
\paragraph{L'atomo di Bohr} La quantizzazione delle orbite elettroniche nel modello atomico di Bohr è fondamentale per la comprensionie degli spettri discreti di emissione e assorbimento. Costituisce anche un primo passo nell'introduzione del dualismo onda-particella, che porterà successivamente alla formalizzazione del concetto di quanto come qualcosa di nuovo rispetto alla fisica classica.

\subsubsection{Curiosità}

Questa sezione sarà dedicata a fenomeni atmosferici spettacolari che possono essere compresi per mezzo della spettroscopia. Lo scopo di questa sezione è di ancorare i concetti discussi precedentemente nella memoria del lettore legandoli a fenomeni con cui ha già familiarità. Protagonisti di questa sezoine saranno i colori delle aurore e dei fenomeni di scarica elettrica, come fulmini, red sprite e ghost.

\chapter{Conclusione}

Il lavoro svolto per questa tesi ha evidenziato significative opportunità di miglioramento per la pagina italiana di Wikipedia dedicata alla spettroscopia. Attraverso un'analisi approfondita delle voci esistenti e la comparazione con le pagine in altre lingue, è emerso chiaramente che la versione italiana presenta notevoli lacune sia in termini di contenuti che di aggiornamento delle informazioni.

Le integrazioni proposte mirano a colmare queste lacune aggiungendo sezioni dettagliate sulla fisica dei fenomeni spettroscopici, esempi di applicazioni pratiche della spettroscopia e curiosità scientifiche come i Transient Luminous Events (TLE) e le aurore boreali. L'inclusione di queste informazioni non solo arricchirà la qualità e la completezza della voce, ma contribuirà anche a stimolare l'interesse e la comprensione della spettroscopia tra i lettori.

L'integrazione di queste informazioni nella pagina di Wikipedia contribuirà a fornire una visione più completa e accurata della spettroscopia, rendendo la pagina una risorsa più utile e affidabile per studenti, ricercatori e appassionati di scienza. Inoltre, questo aggiornamento aiuterà a promuovere la divulgazione scientifica e a valorizzare il ruolo della citizen science, evidenziando come la partecipazione attiva dei cittadini possa arricchire la nostra comprensione del mondo naturale.

\newpage

\printbibliography

\end{document}

