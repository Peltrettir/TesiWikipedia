\documentclass[12pt,a4paper, titlepage, oneside, draft]{article}

\usepackage{geometry}
\usepackage{amsmath}
\usepackage{graphicx}
\usepackage[backend=biber]{biblatex}
\usepackage{hyperref}

%bibliografia
\addbibresource{Spettroscopia.Bibliografia.bib}

%intestazione
\title{Spettroscopia}
\author{Riccardo Peltretti}
\date{\today}

\begin{document}
    \maketitle
    
    \begin{abstract}
        La spettroscopia è la branca della scienza che si occupa di studiare gli spettri elettromagnetici, ovvero le compoenti della luce, prodotti dalle diverse sostanze, per identificarle o osservarne alcune proprietà. L'origine della spettroscopia moderna una branca della scienza che ha origine nel 17° secolo con gli esperimenti di ottica di Isaac Newton (1666-1672), sebbene altri scenziati avessero precedentemente studiato lo spettro solare.\\
        Agli inizi del 19° secolo, grazie agli esperimenti di Joseph von Fraunhofer, la spettroscopia è diventata una tecnica più precisa e scientifica, rivelandosi di fondamentale importanza per la fisica, la chimica e l'astronomia. 
    \end{abstract}

    \tableofcontents
    \clearpage

    \section{Storia}
    Già nella \href{https://it.wikipedia.org/wiki/Naturalis_historia}{\textbf{Historia Naturalis}} di \href{https://it.wikipedia.org/wiki/Plinio_il_Vecchio}{Plinio il Vecchio} troviamo alcuni riferimenti all'esistenza 
    di pietre cerunie, o \href{https://it.wikipedia.org/wiki/Folgorite}{folgoriti}, aventi la capacità di proiettare i colori dell'arcobaleno sulle pareti vicine, quando colpite dalla luce del sole \cite{naturalisHistoria}.\\
    
    \clearpage

    \section{Fisica}
    lorem ipsum
    \clearpage

    \section{Tecniche e Applicazioni}
    lorem ipsum
    \clearpage

    \section{Curiosità}
    lorem ipsum
    \clearpage

    \printbibliography
    \clearpage

    
\end{document}